\documentclass[apj, twocolappendix, numberedappendix, appendixfloats]{emulateapj}
\usepackage[latin2]{inputenc}
\usepackage{amsmath}
\usepackage{graphicx}
\usepackage[pdftex,backref,breaklinks,colorlinks,citecolor=blue]{hyperref}
\usepackage{url}
\pagestyle{plain}
% \usepackage[all]{hypcap}
% \setpapersize{letter}
\usepackage[pass,letterpaper]{geometry}

% \usepackage[title]{appendix}

% \renewcommand\arraystretch{2}
\usepackage{lipsum}

\begin{document}
% \title{A Local In-Situ Stellar Halo Revealed by Gaia}
% \title{In-Situ Stellar Halo in the Solar Neighborhood Revealed by Gaia}
% \title{Gaia Reveals an In-Situ Component of the Stellar Halo in the Solar Neighborhood}
\title{Gaia reveals a metal-rich component of the local stellar halo that predominantly formed in-situ}
% \journalinfo{The Astrophysical Journal, submitted}
\author{Ana Bonaca\altaffilmark{1,5}}
\author{Charlie Conroy\altaffilmark{1}}
\author{Andrew R. Wetzel\altaffilmark{2,3,4,6,7}}
\author{your name here}

\altaffiltext{1}{Department of Astronomy, Harvard University, Cambridge, MA 02138; {ana.bonaca@cfa.harvard.edu}}
\altaffiltext{2}{TAPIR, California Institute of Technology, Pasadena, CA, USA}
\altaffiltext{3}{Carnegie Observatories, Pasadena, CA, USA}
\altaffiltext{4}{Department of Physics, University of California, Davis, CA, USA}
\altaffiltext{5}{ITC Fellow}
\altaffiltext{6}{Moore Prize Fellow}
\altaffiltext{7}{Carnegie Fellow in Theoretical Astrophysics}

\begin{abstract}
We use the first Gaia data release, combined with the ground-based spectroscopic surveys, to search for the origin of kinematically identified halo stars in the Solar neighborhood.
Halo stars, moving at least 220\;km/s with respect to the local standard of rest, are in general more metal-poor than the disk, but surprisingly, half of our halo sample is comprised of stars with $\rm[Fe/H]>-1$.
The orbital directions of these metal-rich halo stars are preferentially aligned with the disk rotation, in sharp contrast with the isotropic orbital distribution of the more metal-poor halo stars.
Similar properties are observed for stars in the Latte simulation of a Milky Way-like galaxy.
In Latte, all of the metal-rich halo stars were formed inside of the solar circle, while the lower metallicity halo has a progressively larger fraction of stars with a more distant origin, out to the virial radius.
This suggests that we have identified a true in situ halo component in the metal-rich halo stars.
Furthermore, their presence in the Solar neighborhood indicates that radial migration has occurred in the Galactic disk.
\end{abstract}
% \keywords{Galaxy: halo --- Galaxy: structure --- cosmology: dark matter}
\maketitle

\section{Introduction}
[IN PROGRESS]
- origin of the stellar halo: lcdm prediction -- accreted, studies at large distances consistent with all accreted
- sims with baryons: also an in situ component, especially in the inner halo/solar radius
- indications for dual halo presence, although no clear origin designations
- halo is old -- important as a window to star formation in the early universe
- in situ reveals early phase of galaxy formation, at a much higher resolution than direct observations of high-z galaxies

- for the first time we can look into detailed properties of a local halo sample
- thanks to Gaia \citep{perryman2001}, DR1 + RAVE-on, we have a large sample of stars with complete phase space info -- can kinematically select an unbiased sample of halo stars
- compare local neighborhood properties to those in latte, a high-res mw-like sim

- confluence of major advances in precision of observations in the solar neighborhood, as well as our theoretical understanding of growth of MW like galaxies
- first driven by Gaia, second in Latte

% Part of the inner halo may have formed in situ, i.e., within the main body of the Galaxy (Abadi et al. 2006). Recent simulations suggest that a fraction of stars formed in the early Galactic disk could have been ejected into the inner halo, and further in situ halo stars could have formed from gas stripped from infalling satellites (Font et al. 2011, McCarthy et al. 2012, Tissera et al. 2013, Cooper et al. 2015, Pillepich et al. 2015), but the quantitative importance of these processes is not yet fully understood. Observationally, evidence for a dual halo has been put forward by Carollo et al. (2007) and Beers et al. (2012), but see Schönrich et al. (2014).

\begin{figure*}
\begin{center}
\includegraphics[width=\textwidth]{../plots/paper/toomre.pdf}
\caption{(Left) Toomre diagram of stars in the Solar neighborhood, from a combined catalog of Gaia--TGAS proper motions and parallaxes, and RAVE-on radial velocities, thus covering the full 6-D phase space.
We kinematically divide the sample into a disk and a halo component.
The halo stars are defined as having $|V-V_{LSR}|>220$\,km/s, and the dividing line is shown in black.
(Right) Positions of TGAS--RAVE-on stars with a measured metallicity in the Toomre diagram.
The color-coding corresponds to the average metallicity of stars in densely populated regions of the diagram, and individual metallicities otherwise.
Interestingly, many halo stars are metal-rich, with $\rm[Fe/H]>-1$.}
\label{fig:toomre}
\end{center}
\end{figure*}


\section{Data}
Studying orbital properties in a sample of stars requires the knowledge of their positions in a 6-D phase space.
Currently, Gaia Data Release 1 (DR1) provides the largest and most precise 5-D dataset for stars in the Solar neighborhood, which we describe in \S\ref{gaia}.
We complement this data with radial velocities from ground-based spectroscopic surveys whose targets overlap with Gaia (\S\ref{rvsurveys}).
Finally, we describe our sample selection in \S\ref{sample}.

\subsection{Gaia}
\label{gaia}
Gaia is the astrometric mission that will map the Galaxy over the next several years \citep{perryman2001}.
The first data from the mission was released in September 2016, and contains not only positions of all Gaia sources ($G<20$), but also positions, parallaxes and proper motions for $\sim$2~million of the brightest stars in the sky \citep{gaiadr1, gaiamission}.
Obtaining the 5-D info after just a year of Gaia's operation was possible by referencing the positions measured with Hipparcos \citep{michalik2015}.
The faintest stars observed by Hipparcos \citep{hipparcos, vleeuwen2007} and released as a part of Tycho~II catalog have $V\sim12$ \citep{hog2000}, which limits the size of the 5-D sample in Gaia DR1 to $\approx2$ million stars.
The joint solution, known as Tycho--Gaia Astrometric Solution \citep[TGAS,][]{gaiaastrometry} is comparable in attained proper motions and parallaxes to the Hipparcos precision (typical uncertainty in positions and parallaxes is 0.3\;mas and 1\;mas/yr in proper motions), but already on a sample that is more than an order of magnitude larger, making TGAS an unrivaled dataset for precision exploration of the Galactic phase space.

\subsection{Spectroscopic surveys}
\label{rvsurveys}
Gaia is measuring radial velocities for $\sim150,000$ stars brighter than $G<16$ \citep{gaiamission}, but the first spectroscopic data will become available only in the second data release.
Thus, we completed the phase-space information of TGAS sources by using radial velocities from ground-based spectroscopic surveys.
We used two distinct spectroscopic datasets, from the RAVE and APOGEE projects, and provide an overview below.

The Radial Velocity Experiment \citep[RAVE,][]{steinmetz2006} is a spectroscopic survey of the southern sky, and its magnitude range $9<I<12$ is well matched to TGAS.
The latest data release, RAVE DR5 \citep{kunder2017}, contains $\sim450,000$ unique radial velocity measurements.
Since RAVE avoided targeting regions of low galactic latitude, the actual overlap with TGAS is $\sim250,000$ stars -- the largest of any spectroscopic survey.
The survey was performed at the UK Schmidt telescope with the 6dF multi-object spectrograph \citep{6df}, in the wavelength range $8410-8795\,\rm\AA$ at a medium resolution of $R\sim7,500$, so the typical velocity uncertainty is $\sim2$\,km/s.
Abundances of up to seven chemical elements are available for a subset of high signal-to-noise spectra.
\citet{casey2016} reanalyzed the RAVE DR5 spectra in a data-driven fashion with The Cannon \citep{ness2015}, providing de-noised measurements of stellar parameters and chemical abundances in the RAVE-on catalog.
In particular, typical uncertainty in RAVE-on abundances is 0.07\;dex, which is at least 0.1\;dex better than precision achieved using the standard spectroscopic pipeline.
Therefore, we opted to use RAVE-on chemical abundances, focusing on metallicities, [Fe/H], and $\alpha$-element abundances.

The Apache Point Observatory Galactic Evolution Experiment (APOGEE) is one of the programs in the Sloan Digital Sky Survey III \citep{majewski2015, sdss3}, which acquired $\sim500,000$ infrared spectra for $\sim150,000$ stars brighter than $H\sim12.2$ \citep{holtzman2015}.
To capitalize on the infrared wavelength coverage, APOGEE mainly targeted the disk plane, but several high latitude fields are included as well \citep{zasowski2013}.
Its higher resolution $R\sim22,500$, provides more precise abundances for a larger number of elements \citep[e.g.,][]{ness2015}.
APOGEE targets are preferentially fainter than stars targeted by RAVE, so its overlap with TGAS is limited to a few thousand stars.
APOGEE and RAVE have different footprints, targeting strategy, and the wavelength window observed, so despite the smaller sample size, we found APOGEE to be a useful dataset for validating conclusions reached by analyzing the larger RAVE sample.

\subsection{Sample selection}
\label{sample}
After matching Gaia--TGAS to the spectroscopic surveys, we increase the precision of the sample by excluding stars with large observational uncertainties.
However, the overlap between TGAS and spectroscopic surveys is limited, and the number density of halo stars in the Solar neighborhood is low.
To ensure that we have a sizeable halo sample, we chose to use very generous cuts on observational uncertainties and propagate them when interpreting our results, rather than restricting our sample size by more stringent cuts.
In particular, we included stars with radial velocity uncertainties smaller than 20\,km/s, and relative errors in proper motions and parallaxes smaller than 1.
In addition, we removed all stars with a negative parallax, to simplify the conversion to their distance.
These criteria select 159,352 stars for the TGAS--RAVE-on dataset, and 14,658 stars for the TGAS--APOGEE sample.

The spatial distribution of stars in our sample is entirely determined by the joint selection function TGAS and the spectroscopic surveys, as we performed no additional spatial selection.
Gaia is an all-sky survey, but since the data is still being acquiring, completeness of the TGAS catalog varies across the sky.
Ground-based spectroscopic surveys have geographically restricted target-list, in addition to the adopted targeting strategy.
This results in a spatially non-uniform sample.
On the other hand, \citet{wojno2016} have shown that the RAVE survey is both chemically and kinematically unbiased.
Thus, focusing on kinematic properties of the sample will result in robust conclusions.

\section{Properties of the local halo stars}
We analyzed properties of halo stars in a sample of bright, $V\lesssim12$, stars within 3\;kpc from the Sun, that have positions, proper motions and parallaxes in the TGAS catalog, and radial velocities from either RAVE or APOGEE.
This sample, though spatially incomplete due to survey selection functions, is kinematically unbiased.
Thus, we define halo kinematically in \S~\ref{sec:sample}, and present its chemical and orbital properties in \S~\ref{sec:chem} and \S~\ref{sec:l}, respectively.

\subsection{Defining a local sample of halo stars}
\label{sec:sample}
Access to the full 6-D phase space information allows us to calculate Galactocentric velocities for all of the stars in the sample.
We summarize the kinematic properties of the sample with a Toomre diagram (Figure~\ref{fig:toomre}), where the Galactocentric $Y$ component on the velocity vector, $V_Y$, is on the x axis, while the perpendicular Toomre component, $\sqrt{V_X^2+V_Z^2}$, is on the y axis. 
This space has been widely used to identify major components of the Galaxy: the thin and thick disks, and the halo \citep[e.g.,][]{venn2004}.
Disk stars dominate a large overdensity at $V_Y\approx220$\;km/s, which corresponds to the circular velocity of the Local Standard of Rest (LSR, $V_{LSR}$).
The density of stars (left panel of Figure~\ref{fig:toomre}) decreases smoothly for velocities progressively more different from $V_{LSR}$, extending all the way to retrograde orbits ($V_Y<0$).

We distinguish between the disk and the halo following \citet{ns2010}: halo stars are identified with a velocity cut $|V-V_{LSR}|>220$\;km/s, where $V_{LSR} = (0,220,0)$\;km/s in the Galactocentric Cartesian coordinates.
The dividing line between the components is marked with a black line in Figure~\ref{fig:toomre}, and both components are labeled in the left panel.
The halo definition employed here is more conservative than similar cuts adopted by previous studies; e.g., \citet{ns2010} defined halo as stars with velocities that satisfy $|V-V_{LSR}|>180$\;km/s.
For example, \citet{sb2009} have shown that the velocity distribution of a Galactic thick disk can be asymmetric, in which case the region $180<|V-V_{LSR}|<220$\;km/s could still contain many thick disk stars.
A higher velocity cut ensures that the contamination of our halo sample with thick disk stars is minimized.
In total, we identified 1,376 halo and 157,976 disk stars, with the halo constituting $\sim1\%$ of our sample.
This is in line with the expectations from number count studies in large-scale surveys \citep[e.g.,][]{juric2008}, although we do not expect an exact match, as TGAS is not volume complete.

The signal-to-noise ratio of 142,086 RAVE-on spectra was high enough to allow a measurement of metallicity [Fe/H].
Alpha-element abundances, [$\alpha$/Fe], were obtained for a subset of 56,259 stars.
Right panel of Figure~\ref{fig:toomre} shows the average metallicity in densely populated velocity bins of the Toomre diagram, while the points in the lower density regions are individually colored-coded by [Fe/H].
As expected, the halo is more metal poor than the disk \citep[e.g.,][]{ivezic2008}.
Within the disk itself, there is a smooth decrease in metallicity further from the VLSR, starting from [Fe/H]$\sim0$ in the thin disk region, $(V_Y, V_{XZ})=(220,0)$\;km/s, to [Fe/H]$\sim-0.5$ in the thick disk region, $(V_Y, V_{XZ})=(100,100)$\;km/s.
Surprisingly, however, there are many stars with thick disk-like metallicities found in the halo region of the Toomre diagram, and some of them are on very retrograde orbits.
In the following sections we explore the chemical composition (\S\ref{sec:chem}) and orbital properties of these stars (\S\ref{sec:l}).

\subsection{Chemical composition}
\label{sec:chem}
In this section we study the chemical composition of the Solar neighborhood stars observed by both Gaia-TGAS and RAVE.
Figure~\ref{fig:mdf} (top) shows the metallicity distribution for the two kinematic components identified above: the disk in red and the halo in blue.
The disk is more metal rich than the halo, and peaks at the approximately solar metallicity, $\rm[Fe/H]=0$.
The halo is more metal poor, and exhibits a peak at $\rm[Fe/H]\sim-1.6$, typical of the inner halo \citep[e.g.,][]{allende-prieto2006}.
However, the metallicity distribution of the halo has an additional peak at the metal-rich end, centered on $\rm[Fe/H]\sim-0.5$ and extending out to the super-solar values.

To corroborate the existence of metal-rich stars on halo orbits, we also show the metallicity distribution function for TGAS stars observed with APOGEE at the bottom of Figure~\ref{fig:mdf}.
The disk--halo decomposition for the APOGEE sample was performed in the identical manner to that of RAVE-on.
The metallicity distributions between the two surveys are similar: the disk stars are metal-rich, while the halo has a wide distribution ranging from $\rm[Fe/H]\sim-2.5$ to $\rm[Fe/H]\sim0$.
The bimodality in the metallicity distribution of APOGEE halo stars, although less obvious than in the RAVE-on sample due to the smaller sample size, is still present, with $\rm[Fe/H]\approx-1$ separating the two peaks.
At a closer inspection, the apparent bimodality in the metallicity distribution of RAVE-on halo stars is slightly more metal-poor, $\rm[Fe/H]\approx-1.1$, than observed in the APOGEE sample, $\rm[Fe/H]\approx-0.8$.
In what follows, we compromise between these two values, and split the halo sample at $\rm[Fe/H]=-1$, into a metal-rich ($\rm[Fe/H]>-1$) and a metal-poor component ($\rm[Fe/H]\leq-1$).

\begin{figure}
\begin{center}
\includegraphics[width=0.9\columnwidth]{../plots/paper/mdf.pdf}
\caption{Metallicity distribution function of the Solar neighborhood in TGAS and RAVE-on catalogs on the top, and TGAS and APOGEE at the bottom.
Kinematically-selected disk stars are shown in red, while the halo distribution is plotted in blue.
In both samples, there is a population of metal-rich halo stars, with $\approx50\%$ of stars having $\rm[Fe/H]>-1$ (marked with a vertical dashed line).}
\label{fig:mdf}
\end{center}
\end{figure}

\begin{figure}
\begin{center}
\includegraphics[width=0.9\columnwidth]{../plots/paper/afeh.pdf}
\caption{Chemical abundance pattern, [$\alpha$/Fe] vs [Fe/H], for TGAS--RAVE-on sample on the top and TGAS--APOGEE at the bottom.
The pattern for disk stars is shown as a red-colored Hess diagram (logarithmically stretched), while the halo stars are shown individually as blue points.
In both surveys, the metal-poor halo is $\alpha$-enhanced, while the metal-rich halo follows the abundance pattern of the disk.}
\label{fig:afeh}
\end{center}
\end{figure}

Chemical abundances have been used to discern different components of the Galaxy \citep[e.g.,][]{galrev1998}.
The abundance space of [$\alpha$/Fe] vs [Fe/H] is particularly useful in tracing the origin of individual stars \citep[e.g.,][]{lee2015}.
Figure~\ref{fig:afeh} shows this space for RAVE-on spectra on the top, and APOGEE on the bottom.
The disk distribution is shown as a red density map, while the less numerous halo stars are shown individually in blue.
Similarly to the overall metallicity distribution function, RAVE-on and APOGEE surveys are in a qualitative agreement in terms of the more detailed chemical abundance patterns as well.
At low metallicities, the halo is $\alpha$-enhanced, but at high metallicities its [$\alpha$/Fe] declines, following the disk abundance pattern, both in terms of the mean [$\alpha$/Fe] and its range at a fixed [Fe/H].
In particular, thick disk stars have higher $\alpha$-element abundances at a given metallicity \citep[e.g.,][]{nidever2014}, and follow a separate sequence visible in the more precise APOGEE data ([Fe/H]$\sim-0.2$, [$\alpha$/Fe]$\sim0.2$), while the thin disk is in general more metal-rich and $\alpha$-poor.
Metal-rich halo stars in both samples span this range of high- and low-$\alpha$ abundances.
At lower metallicities, \citet{ns2010} reported that halo stars follow two separate [$\alpha$/Fe] sequences, with the high-$\alpha$ stars being on predominantly prograde orbits, whereas the low-$\alpha$ stars are mostly retrograde.
We do not resolve the two sequences, or see any correlation between the orbital properties and [$\alpha$/Fe] at a fixed [Fe/H].
However, the RAVE-on abundances are fairly uncertain (typical uncertainty is $\sim0.07$\;dex, marked by a black cross on the top left in Figure~\ref{fig:afeh}), so the sequences seen in the higher-resolution data from \citet{ns2010} would not be resolved in this data set.
In the APOGEE sample, where the typical uncertainty is smaller, $\sim0.03$\;dex, the number of halo stars is too small to unambiguously identify multiple $[\alpha/\rm Fe]$ sequences.

The evidence presented so far shows that the stellar halo in the Solar neighborhood has a metal-poor and a metal-rich component.
Given how the metal-rich component follows the abundance pattern of the disk, we discuss the possible contamination by the thick disk in Appendix~\ref{sec:tdcontamination}, and rule out the possibility that the whole of the metal-rich halo is attributable to the canonical thick disk.
In the next section, we proceed to characterize the orbital properties of the two halo components.

\begin{figure}
\begin{center}
\includegraphics[width=\columnwidth]{../plots/paper/ltheta.pdf}
\caption{Orientation of angular momenta with respect to the Galactocentric $Z$ axis for different Galactic components: the disk in red, metal-poor halo in dark blue and metal-rich halo in light blue.
The angular momenta of disk stars are aligned with the $Z$ axis, while those of halo stars are more uniformly distributed.
There is an excess of metal-rich halo stars on prograde orbits, $\theta_L>90^\circ$, compared to the metal-poor halo orbital orientations.}
\label{fig:ltheta}
\end{center}
\end{figure}

\subsection{Angular momenta}
\label{sec:l}
Stellar orbits can be classified in terms of their integrals of motion, however, calculating these requires the knowledge of the underlying gravitational potential \citep{bt2008}.
Furthermore, in a realistic Galactic environment, some stars are on chaotic orbits \citep[e.g.,][]{price-whelan2016}, where the integrals of motion do not exist.
On the other hand, any star with a measured position in a 6-D phase space has a well defined angular momentum.
In this section we use angular momenta as empirical diagnostics of stellar orbits, and focus in particular on the orientation of the angular momentum vector with respect to the Galactocentric $Z$ axis, quantified by the angle
\begin{equation}
\theta_L \equiv \arctan(L_Z/|\vec{L}|)
\label{eq:thetal}
\end{equation}
where $L_Z$ is the $Z$ component of the angular momentum, and $|\vec{L}|$ its magnitude.
$L_Z$, and hence $\theta_L$, are conserved quantities in static, axisymmetric potentials, such as that of the Milky Way disk.
This has already been utilized to identify coherent structures in the phase space of local halo stars \citep[e.g.,][]{helmi1999, smith2009}.
In the adopted coordinate system, the disk orientation angle is $\theta_L=180^\circ$, so that prograde orbits are those with $\theta_L>90^\circ$ and retrograde have $\theta_L<90^\circ$.
We show the distribution of angular momentum orientations $\theta_L$ for the identified Galactic components in Figure~\ref{fig:ltheta}.

\begin{figure*}
\begin{center}
\includegraphics[width=\textwidth]{../plots/paper/toy_model.pdf}
\caption{Toy model for the phase space of the solar neighborhood.
The model consists of a halo (blue) and a disk component (red), with their positions drawn directly from the TGAS--RAVE-on sample, and kinematics from the velocity ellipsoids of \citet{bensby2003}.
Left panel shows the model in the Toomre diagram.
The black line is the employed demarcation between the halo and the disk, which does fairly good job in separating the two in the toy model as well.
The central panel shows the orientation of angular momenta in the model, with each component shown as a shaded histogram, and a model total with a black line.
Toy model angular momenta successfully reproduce the angular momentum orientations observed in the Milky Way (gray dashed line).
Kinematically selecting the halo in the model (right panel, shaded histogram) produces a distribution in excellent agreement with the distribution of metal-poor halo stars in the Milky Way (dark blue line).
The metal-rich halo in the Milky Way (light blue line) is inconsistent with being a part of an isotropic halo studied in this toy model.}
\label{fig:toy}
\end{center}
\end{figure*}

As expected, most of the disk stars are indeed on orbits in the disk plane with $V_Z\approx0$, and have $\theta_L\approx180^\circ$ (red histogram in Figure~\ref{fig:ltheta}).
Angular momenta of both halo components span the entire range of $0^\circ<\theta_L<180^\circ$, but in detail, their distributions are significantly different from each other.
The metal-poor halo has almost a flat distribution as a function of $\theta_L$, with a slight depression at very prograde angles (dark blue histogram in Figure~\ref{fig:ltheta}).
The metal-rich halo is predominantly prograde, but also has a long tail to retrograde orbits (light blue histogram in Figure~\ref{fig:ltheta}).
In the next section, we explore the origin of these distributions in terms of a toy model for the kinematic distribution of the Galaxy, as well as in comparison to a hydrodynamical simulation of a Milky Way-like galaxy.

\section{Origin of halo stars in the Solar neighborhood}
The three main components of the Galaxy present in the solar neighborhood are the thin disk, the thick disk and the halo \citep[e.g.,][]{bhg2016}.
In this section we test whether the measured orientations of the angular momenta (\S\ref{sec:l}) are well described in the context of such a three component model, or whether there are additional features revealed by Gaia.
We first study the distribution of angular momenta using a toy model of the Galaxy in \S\ref{sec:toymodel}, and then analyze properties of a solar-like neighborhood in a cosmological hydrodynamical simulation (\S\ref{sec:latte}).

\subsection{Toy model}
\label{sec:toymodel}
To construct a toy model for our TGAS--RAVE-on sample, we assign stars to one of the three components considered: a thin disk, a thick disk, or a halo.
The model is defined by the number of stars in each component, their spatial and kinematic properties.
To set the number of halo stars in the toy model, we assumed that the halo is isotropic.
In that case, an equal number of halo stars are on prograde and retrograde orbits. 
Since we expect no contamination from the disk on retrograde orbits, we set the total size of the halo in the toy model to be twice the number of retrograde stars in our Milky Way sample.
For the remaining disk stars, we vary the ratio of thin to thick disk stars to best match the distribution of prograde orbits.

Once the number of stars in each component had been determined, we proceeded to assign them their phase space coordinates.
Our sample is spatially confined to within only several kpc from the Sun, so we see no differences in spatial distributions of the kinematically defined components from section \S\ref{sample}.
This allowed us to take the spatial distribution of stars in our sample, and randomly designate them a component in the toy model, thus ensuring that the spatial selection function of both TGAS and RAVE-on is properly reproduced in the model.
For each star in the model, we drew a 3-D velocity from its component's velocity ellipsoid measured by \citet{bensby2003} on a smaller sample of local stars with Hipparcos parallaxes and proper motions, and which accounts for the asymmetric drift.
With positions and velocities in place, we calculated the angular momenta and their orientation angles with respect to the $Z$ axis, $\theta_L$, for all stars in the toy model.

The properties of our toy model are summarized in Figure~\ref{fig:toy}.
Left panel shows the components of the model in the Toomre diagram, with the disk stars in red, halo stars in blue, and the black line delineating our kinematic boundary between the halo and the disk.
The velocity ellipsoids of these components overlap, so our kinematic definition of a halo produces a sample which is likely neither pure, nor complete.
This is illustrated in the toy model by several thick disk stars that enter the halo selection box at $(V_Y, V_{XZ}) \simeq (100,200)$\;km/s, and also halo stars on prograde orbits, which fall outside of the halo selection.
As no simple kinematic cut will completely separate the different components, we opted to emphasize the purity of our halo sample.
Based on the toy model, we estimate that the fraction of interloping disk stars in a kinematically defined halo is only $\sim10\%$, but this in turn makes the halo sample less complete at $75\%$.
For a comparison, \citet{ns2010} defined a halo that is $\sim90\%$ complete, but only $\sim55\%$ pure.

\begin{figure}
\begin{center}
\includegraphics[width=0.9\columnwidth]{../plots/paper/latte_mwcomp.pdf}
\caption{Star particles from the Milky Way-like hydrodynamical simulation Latte, observed identically to stars in the Solar neighborhood.
Left panel shows positions of star particles in the Toomre diagram, color-coded by metallicity.
Metallicity distribution function of Latte particles are in the middle, with disk being shaded red, and halo blue.
On the right, we show the orientations of Latte angular momenta, with disk in red, metal-poor halo in dark blue and metal-rich halo in light blue.
Properties of simulated star particles are remarkably similar to the distributions of stars in our Milky Way sample (reproduced as empty histograms on the middle and right panel for direct comparison).}
\label{fig:latte}
\end{center}
\end{figure}

The middle panel of Figure~\ref{fig:toy} shows the orientation of angular momenta, $\theta_L$, for the components of the toy model, with halo in blue and disk in red.
As expected, the majority of disk stars are moving in the disk plane, and the distribution is sharply peaked at $\theta_L=180^\circ$.
The nearly isotropic halo of the \citet{bensby2003} model has a flat distribution in $\theta_L$.
The sum of the two components is represented with a thick black line, which compares favorably to the distribution of $\theta_L$ observed in the Milky Way, and shown in dashed gray.
The agreement between the toy model and the Milky Way is particularly good at very prograde and very retrograde orbits.
The abundance of stars on only slightly prograde orbits ($\theta_L\approx100^\circ$) is somewhat underestimated in the toy model, indicating that the transition between a disk and a halo component in the Milky Way is more gradual than what can be reproduced by a simple model featuring only two disks and an isotropic halo. %captured

The right panel of Figure~\ref{fig:toy} compares the modeled halo and the observed one.
For a fair comparison, in this panel we only consider model stars that would satisfy our kinematic halo selection, which are a combination of some disk and the majority of halo stars, as discussed above.
The distribution of $\theta_L$ for this kinematically-selected halo stars from a toy model is shown as a shaded histogram.
While the intrinsic distribution of an isotropic halo is flat (middle panel), kinematic selection introduces a suppression at the most prograde orbits (right panel).
Kinematic selection excludes halo stars with $|V-V_{LSR}|\leq220$\;km/s, all of which are on prograde orbits, which is manifested as a depression at $\theta_L\gtrsim90^\circ$.
The magnitude of this depression exactly matches the distribution of metal-poor halo stars in the Milky Way, overplotted with a dark blue line in the right panel of Figure~\ref{fig:toy}.
This suggests that the metal-poor halo in the Solar neighborhood is intrinsically isotropic, and that all stars more metal-poor than $\rm[Fe/H]\leq-1$ are likely a part of the metal-poor halo, even if they are kinematically consistent with the disk.
The distribution of metal-rich halo stars in the Milky Way, shown as light blue line in the righ panel of Figure~\ref{fig:toy}, shows the opposite behavior of an excess at prograde orbits and is significantly different from the toy model prediction for an isotropic halo.

A simple toy model successfully explains bulk properties of our sample: most of the stars are in a rotating disk, with a minority in an isotropic halo, which maps well to the metal-poor halo stars identified in our TGAS--RAVE-on sample.
It also points out that the metal-rich halo stars appear to be in a transitional stage between the prograde disk, and the isotropic halo distribution.
Next, we analyze the origin of such a population.

\subsection{The Latte simulation}
\label{sec:latte}
Numerical simulations have long been used to interpret the observed properties of galaxies and uncover the underlying physical mechanisms governing their evolution.
In the $\Lambda$CDM cosmology, stellar halos are naturally produced by disruption of the accreted satellite galaxies \citep[e.g.,][]{bj2005, johnston2008}, although more realistic simulations have shown that a fraction of halo stars can be formed in situ, especially in the inner parts of the galaxy \citep[e.g.,][]{zolotov2009}.
The exquisite resolution of the latest generation of hydrodynamical simulations now allows for an almost one-to-one comparison between individual stars observed in the Milky Way and simulated star particles.
In this section, we examine a Milky Way-like halo from the Latte simulation suite \citep{wetzel2016}, part of the FIRE project, with the goal of understanding the origin of halo stars in the Solar neighborhood.

The Latte simulations are a suite of Milky Way-sized galaxies simulated at an unprecedented spatial and temporal resolution using the GIZMO code \citep{hopkins2015} with FIRE-2 implementation of star formation and stellar feedback \citep[Feedback In Realistic Environments,][]{hopkins2017}.
This setup has been used to simulate galaxies ranging in mass from isolated ultra-faint dwarfs \citep{wheeler2015} up to the Milky Way-like galaxies \citep{wetzel2016}.
These studies successfully reproduced the observed internal properties of dwarf galaxies \citep{elbadry2016}, morphologies of disk galaxies \citep{ma2016}, and the satellite population around Milky Way-sized galaxies \citep{wetzel2016}.
In addition, galaxies formed in these zoom-ins lie on the global scaling relations of the observed galaxy populations \citep{hopkins2014, feldmann2016}.
In addition, all of the numerical choices and physical inputs were extensively tested \citep{hopkins2017}.
Major improvement over previous practices relevant for this work was the implementation of metal diffusion, a process which results in a more realistic distribution of metals in the interstellar medium, and hence the resulting abundance patterns.
In the rest of this section, we analyze the highest-resolution Milky Way-sized galaxy available with metal diffusion: Latte halo m12i, simulated with particle masses x (y, z) and force resolution x (y, z) for stars (gas, dark matter).

To construct a sample of Latte particles analogous to the TGAS--RAVE-on sample, we first aligned the simulation coordinate system with the disk, and then selected star particles in a 3\;kpc sphere located at a distance of 8.3\;kpc from the center of the galaxy.
We classified Latte particles as either disk or halo using a kinematic cut in the Toomre diagram similar to the one employed for stars in the Milky Way (\S\ref{sec:sample}).
Since the circular velocity in Latte is slightly different from the Galactic value, the definition of the local standard of rest is also different, but we keep the same conservative measure for the dispersion of 220\;km/s in the Latte disk.
The Toomre diagram for Latte sample is shown in the left panel of Figure~\ref{fig:latte}.
Qualitatively, it is similar to that of the Milky Way (Figure~\ref{fig:toomre}), with most of the star particles rotating in the disk at $\sim235$\;km/s, and the density of stars smoothly decreasing away from the local standard of rest.
Quantitatively, the halo fraction is an order of magnitude higher at 10\%, and, compared to the Milky Way's, its kinematic space is more structured.
This is partly due to there being no selection effects in the Latte sample, so it effectively extends to larger distances, where we expect the halo to constitute a larger mass fraction.
Any kinematic structures are hence better sampled and more readily observable in Latte.
Additionally, the Latte sample is free of uncertainties, and at least some of the structure present in the Milky Way sample is smoothed by the observational uncertainties \citep[see, e.g.,][]{sanderson2015}.

Star particles in the Toomre diagram of Figure~\ref{fig:latte} are color-coded by metallicity, and show trends similar to those observed in the Milky Way.
For a more quantitative analysis, we show metallicity distribution of the Latte disk and halo components in the central panel of Figure~\ref{fig:latte}.
The Latte halo is more metal poor than its disk, and although there is no bimodality in the halo metallicity, the whole distribution is as wide as the one observed in the Milky Way, extending from [Fe/H]$\lesssim-2$ to [Fe/H]$\simeq0$.
This agreement, in addition to $\rm[\alpha/Fe]$ abundance trends recovered in simulated disks (Wetzel et al., in prep), certifies that feedback processes invoked in FIRE capture the essence of chemical evolution in galaxies.
Same as in the Milky Way sample, we proceed to divide the Latte halo in a metal-rich, [Fe/H]$>-1$, and a metal-poor component, [Fe/H]$\leq-1$.
The ratio of metal-rich to metal-poor halo stars in Latte is not quite the same as in the Milky Way, but this does not seriously impede our goal of qualitatively understanding differences between the two populations.

Finally, we analyze orbital properties of Latte star particles by showing the orientation of their angular momenta with respect to the $Z$ axis, $\theta_L$, in the right panel of Figure~\ref{fig:latte}.
The angular momenta of Latte disk particles are well aligned with the $Z$ axis, with $\theta_L\simeq180^\circ$, similar to disk stars in the Milky Way.
The Latte halo shows a flatter distribution of $\theta_L$, but there is still an excess of metal-rich particles on prograde orbits (light blue histogram) with respect to the metal-poor halo particles (dark blue).
Overall, Latte star particles have similar kinematic, chemical and orbital properties to stars observed in the Milky Way, indicating that the galaxy created in this Latte halo could also share the formation history with the Milky Way.
We next trace Latte star particles back to their birthplace and suggest a possible scenario for the formation of the Solar neighborhood.

\subsection{Formation scenario}
We define the origin of a Latte star particle as its distance from the host galaxy at formation time, and inspect this formation distance as a function of a particle's age in Figure~\ref{fig:dform}.
Particles with disk kinematics are shown in red, while those identified as halo are blue.
Globally, there is a suppression of particles reaching the solar neighborhood that were formed $\sim$7\;Gyr ago, which coincides with the time of the last major merger ($z\sim0.7$).
This event brought a large amount of gas to the center of the galaxy, which switched the star-forming conditions from those producing mainly stars on halo-like orbits (more than 95\% of halo particles are older than 6\;Gyr) to the orderly production of disk particles (more than 90\% of disk particles were formed in the last 6\;Gyr).
The formation distances between the disk and the halo component are equally dichotomous: most of the disk stars were formed close to their present-day distance from the galactic center ($5-11$\;kpc, shaded gray in Figure~\ref{fig:dform}), while the halo particles originate from the extremes of the central 1\;kpc, to the fringes outside of the virial radius.
Overall, only $\sim20$\% of Latte star particles on halo orbits were formed inside of their present-day radial distance range, indicating that radial migration is an important phenomenon sculpting the inner Galaxy.
We discuss the implications of the halo component undergoing radial migration in \S\ref{sec:migration}.

\begin{figure}
\begin{center}
\includegraphics[width=\columnwidth]{../plots/paper/latte_dform2.pdf}
\caption{Distance from the Latte host galaxy at the time of formation as a function of stellar age.
Gray shaded region marks the  present-day position in the solar neighborhood.
Blue circles are kinematically identified as halo (with metal-rich particles plotted in light blue, and metal-poor are in dark blue), red circles are kinematically consistent with the disk.
The galaxy was undergoing a major merger 7\;Gyr ago, so most stars formed at that time come from the inner 1\;kpc. 
We define origin of a star as being accreted if it formed more than 20\;kpc from the host galaxy (horizontal black line), and in situ otherwise.
Disk stars have been formed in situ and at later times, while the halo formed early, and has a population of both accreted and in situ particles, with the accreted component being more metal-poor.}
\label{fig:dform}
\end{center}
\end{figure}

The formation distance of a star particle can also be used as a rudimentary diagnostic of the formation mechanism.
From Figure~\ref{fig:dform} we see that almost none of the disk particles were formed beyond 20\;kpc (indicated with a horizontal black line), so we adopt this distance as a delimiter between the in situ and accreted origin for star particles in Latte.
Most of the accreted particles are classified as halo, and were formed inside dwarf galaxies merging with the Latte host.
The tracks in the space of formation distance and age (Figure~\ref{fig:dform}) delineate the orbits of these dwarfs.
All of the satellites are disrupted once they get within the central 20\;kpc, and for most of them this happens on the first approach.
In the process, they bring in gas which fuels in situ star formation.
At early times, most of the particles formed in situ also become a part of the halo, while the last major merger starts the onset of the in situ disk formation.
Although satellite accretion onto the Latte host galaxy continues after the last major merger, none of the accreted particles reach the solar circle, so effectively all of the halo particles in the solar neighborhood were formed prior to the last major merger.
This is in line with findings of \citet{zolotov2009}, who showed that late-time accretion predominantly builds outer parts of the stellar halo.

In summary, at late times, Latte particles are being formed in the disk, while the old Latte particles are predominantly a part of the halo.
A third of the halo in the solar neighborhood was accreted from the infalling satellites, while the majority of the particles were formed in situ in the inner galaxy and migrated to the solar circle.
In this simplified depiction for the origin of the stellar halo, we do not account for the possibility that the particles formed within 20\;kpc could have still been bound to a satellite galaxy, and hence accreted, nor do we distinguish between different modes of in situ halo formation (e.g., through a dissipative collapse; \citealt{samland2003}, or with stars being heated from the disk; \citealt{purcell2010}).
These are not serious shortcomings, as satellites that entered the inner 20\;kpc were disrupted soon thereafter, so any newly-formed particles would have been only loosely bound to the satellite at the time.
Furthermore, given that the median formation distance of the in situ halo is $\sim4$\;kpc, we expect that dissipative collapse only marginally contributes to the census of halo particles at the solar circle, but ultimately we draw conclusions which are insensitive to the particulars of their origin.
We next consider the origin of Latte star particles in terms of our observables -- their metallicity and angular momentum orientation.

The average fraction of accreted particles in the Latte halo is presented as a function of metallicity and the orientation of the angular momentum vector, $\theta_L$, in Figure~\ref{fig:facc}.
There is a very good correlation between metallicity and the accreted fraction, with all of the lowest metallicity particles, [Fe/H]$\lesssim-2.5$, having been accreted (yellow), and all of the metal rich halo particles, [Fe/H]$\gtrsim-0.5$, having formed in situ (purple).
The accreted fraction shows almost no dependence on the angular momentum orientation of a particle, making metallicity a good diagnostic of a particle's origin.
Combined with the similarities between the global chemical and orbital properties in Latte and the Milky Way, this result suggests that the metal-rich halo component identified in the TGAS--RAVE-on sample has been formed in the inner Galaxy and driven to the Solar circle through subsequent secular evolution.
In the next section we outline how to test this hypothesis with the near-future data (\S\ref{sec:ages}), and discuss broader implications that the proposed scenario for the formation of the Solar neighborhood implies for the formation of the Galaxy (\S\ref{sec:diskheating}).

\begin{figure}
\begin{center}
\includegraphics[width=\columnwidth]{../plots/paper/latte_facc.pdf}
\caption{Average fraction of accreted stars in the solar neighborhood of the Latte galaxy as a function of metallicity and angular momentum orientation angle.
A star's origin has almost no dependence on its current orbital properties, but it correlates very well with its metallicity.
The accreted fraction is varying smoothly between the metal-rich end where all of the stars have been formed in situ, and the metal-poor end where all stars have been accreted.
Thus, in the inner halo, stellar metallicity is a better indicator of its origin than kinematics.
}
\label{fig:facc}
\end{center}
\end{figure}

\section{Discussion}

In the previous section, we identified halo metallicity as a good indicator of its origin, and concluded that the metal-rich halo stars we identified kinematically in the Solar neighborhood have formed in situ, inside the Solar circle.
Here we review previously reported evidence for metal-rich halo stars in \S~\ref{sec:previous}, discuss mechanisms which heat stars originally formed in the disk to halo-like orbits in \S~\ref{sec:diskheating} and suggest a test for our conclusions by analyzing stellar ages as a function of metallicity in \S~\ref{sec:ages}.

\subsection{Previous evidence for a metal-rich halo}
\label{sec:previous}
Large-scale spectroscopic surveys have mapped the distribution of stellar metallicity in the Milky Way \citep[e.g.,][]{ivezic2008}.
Some of these stars have distance estimates, and can therefore be spatially identified as a part of the disk or a halo.
Relevant for this discussion, halo stars in the outer Galaxy, $R_{G}>15$\;kpc, are very metal-poor, with median $\rm[Fe/H]\sim-2.2$, while in the inner Galaxy, typical metallicity of a halo star is $\rm[Fe/H]\sim-1.6$ \citep[e.g.,][]{carollo2007, dejong2010}, which agrees well with the metal-poor halo component identified in our sample.
Also similar to our findings, metallicities of the inner halo extend to the solar value, even though these are the tail of the metallicity distribution \citep[e.g.,][]{allendeprieto2006}, and not an additional component that is visible in Figure~\ref{fig:mdf}.

However, employing only spatial information makes it hard to distinguish between the halo and the thick-disk, especially at high metallicity.
The difference between the thick disk and the halo is more pronounced in kinematics, so \citet{sheffield2012} selected outliers from the disk velocity field to study the halo.
They identified eight metal-rich halo stars, whose $\alpha$-abundances are consistent with disk, and suggested these have been kicked out of the disk.
Still, none of these stars have kinematics that could rule out a thick disk interpretation at a $3\;\sigma$ level.

To date, only one metal-rich star has a high probability of being a halo star \citep{hawkins2015}.
This star has high velocity in the Galactic rest frame, $V_{GSR}\simeq430$\;km/s, and is on an eccentric orbit, $e=0.72$, that reaches $\sim30$\;kpc above the Galactic plane, but has a metallicity $\rm[Fe/H]=-0.18$, so \citet{hawkins2015} concluded it has been ejected from the thick disk.
Though valuable as an indicator of the processes governing the assembly of the Galaxy, a single star 

In this paper, we presented the first unambiguous detection of a significant population of metal-rich halo stars.
In the next couple of sections we explore potential origin scenarios and implications for the upcoming observables.

\subsection{Disk heating mechanisms}
\label{sec:diskheating}
The excess of metal-rich halo stars on prograde orbits indicates they originate from within the Milky Way disk, but it is unclear at what Galactocentric distance these stars were formed.
The bulk of similarly selected Latte particles originate from the inner Galaxy, implying a degree of radial migration occurred.
Alternatively, these stars could be runaway stars -- stars kicked out of the disk at the Solar radius during binary interactions, which is a process not captured within the Latte simulation.
In this section we explore implications of these opposing disk heating mechanisms, and assess how plausible they are in explaining metal-rich halo stars detected in RAVE-on--TGAS.

\subsubsection{Runaway stars}
\label{sec:runaway}
Runaway stars are young, usually OB stars that were formed in the disk and ejected from their birthplace \citep{blaauw1961}.
Some of them have been found in the halo \citep[e.g.,][]{conlon1990}, so it is sensible to test whether any of our metal-rich halo stars are in fact runaway stars.
Having formed in the disk recently, runaways should have high metallicities, and \citet{bromley2009} have already suggested that solar-metallicity stars reported at 5\;kpc away from the Galactic place \citep{ivezic2008} could be runaway stars.
The metallicity distribution of our metal-rich halo peaks at $\rm[Fe/H]\approx-0.5$, so most of them are probably not runaway stars.
However, the high-metallicity tail of our halo sample extends to super-solar values, so we test whether any of those stars are consistent with being runaways.

To test the runaway hypothesis, we analyze the fraction of runaways expected in a disk population.
Recently, \citet{bromley2009} studied the dynamical evolution of stars ejected from the disk via the massive binary mechanism, while \citet{perets2012} performed a similar study assuming runaways originate from two-body interactions in dense clusters.
Both studies recovered the observation that a fraction of runaways increases with stellar mass: up to 40\% of O stars are displaced from their birthplace, while this fraction is 5\% for B stars \citep{blaauw1961, gies1986}, and for the first time estimated that the fraction of runaway A stars is $\approx2\%$.
There are no OB stars in our metal-rich halo sample, most likely because performing spectroscopic analysis of these hot stars is challenging, so none were included in the RAVE-on catalog.
There are, however, three A halo stars in our sample, and all of them have a super-solar metallicity, $\rm[Fe/H]>0$.
These are prime candidates for runaway stars, but they constitute only 0.5\% of the metal-rich halo sample, so we can safely conclude that runaway stars are a minor component of the observed metal-rich stellar halo.

\subsubsection{Radial migration}
\label{sec:migration}
In the classical picture of radial migration, spiral structures in the disk radially scatter nearby stars up to several kpc \citep{sellwood2002}.
Stars that have migrated outwards are typically more metal-rich than their neighbors, so radial migration explains why some stars in the Solar neighborhood (including the Sun!) are more metal-rich than both the surrounding stellar population and the local interstellar medium \citep{wielen1996}.
Idealized simulations of disk formation have found radial migration to be a key component in explaining numerous observables, such as the spread in the age--metallicity relation \citep{roskar2008} or the disk morphology and its abundance patterns \citep{schonrich2009}.
Such studies have also shown that stars migrating outwards reach larger heights above the disk plane \citep[e.g.,][]{schonrich2009, loebman2011}.
This scenario can explain, at least in part, formation of the thick disk \citep[e.g.,][]{wilson2011}, so it is also conceivable that the metal-rich stars we identified on the halo-like orbits are an endpoint of this process.
However, subsequent numerical works have found that outward migrators do not attain larger scale heights from the disk plane \citep{minchev2012, vera-ciro2014}, thus casting doubts on the idea of forming the thick disk and a metal-rich halo through radial migration.

However, these simulations have been initialized to match the orderly morphology of stellar disks observed at the present day, and do not capture the range of dynamical conditions present in the cosmological simulations of galaxy formation \citep[e.g.,][]{agertz2009}.
In particular, the metal-rich halo in Latte was formed while the host galaxy was still actively accreting.
The merging subhalos brought in a lot of gas to the galactic center (evident as star-forming tracks in Figure~\ref{fig:dform}), which fueled additional in situ star formation.
\citet{elbadry2016} showed that these conditions create two mechanisms for radially displacing stars.
First, some stars are formed in the large gas flows, so their orbits can be rather eccentric and have large apocenters.
Second, the combination of inflowing gas from mergers and outflows from star formation feedback produce strong fluctuations in the underlying gravitational potential.
Such fluctuations have already been shown to change the distribution of dark matter particles \citep[e.g.,][]{pontzen2012, brooks2014, dicintio2014}, but \citet{elbadry2016} showed that stellar orbits are affected as well, ultimately becoming heated to a nearly isotropic distribution.
This mechanism is most efficient in relatively shallow potential wells of dwarf galaxies.
At the present day, Latte host galaxy is too massive to exhibit such behavior, but its progenitor was much less massive while the metal-rich halo was being formed, so it is likely that similar processes drove their radial migration to the Solar circle.

Radial migration, driven by large-scale motions in the Milky Way progenitor, could explain the origin of metal-rich stars on halo-like orbits in the Solar neighborhood.
If these stars truly originate from the inner Galaxy, then they not only illustrate an important dynamical mechanism shaping the Galaxy, but are also a unique window into star formation in the early Milky Way.
In the next section, we propose to test this origin scenario using stellar ages.

\begin{figure}
\begin{center}
\includegraphics[width=0.9\columnwidth]{../plots/paper/latte_ages.pdf}
\caption{Metallicity range (16--84 percentile) of star particles as a function of their age for three structural components identified at the solar circle of the Latte simulation: disk (red), in-situ halo (light blue) and accreted halo (dark blue).
The in-situ halo follows the metallicity evolution of the disk, while the accreted halo particles of the same age are consistently more metal poor.
This prediction can be directly tested once stellar ages are available for the Gaia stars.}
\label{fig:ages}
\end{center}
\end{figure}

\subsection{Inferring the halo origin with stellar ages}
\label{sec:ages}
Ages of individual stars are an important diagnostic of their origin.
For example, runaway stars are expected to be younger than those dynamically dispersed from the inner Galaxy, so measuring the ages of metal-rich halo stars identified in this study could directly distinguish between these two origin scenarios.
When combined with other observables, such as kinematics and composition, stellar ages can also illuminate dynamical processes operating in the Galaxy.
In this section we explore correlations between metallicity and age for stars in the Solar neighborhood, as predicted by the Latte simulation.

Figure~\ref{fig:ages} shows how the metallicity of Latte star particles depends on their age for disk (red), in-situ halo (light blue) and accreted halo (dark blue).
Shaded regions correspond to the 16th to 84th percentile in the distribution of metallicities for star particles of a given age.
In general, metallicity increases with time, however, accretion of a significant amount of the low metallicity gas during the last major merger 7\;Gyr ago is evident as a decrease in the metallicity of stars formed in situ immediately following this event.
Comparing different Latte components, we note that the metallicities of halo particles formed in situ closely follow the evolution of disk particles, while the accreted halo is more metal poor at all ages.
The bifurcation in the metallicity tracks for the in-situ and accreted halo is a prediction of the Latte simulation, which, if confirmed observationally, can be used to directly differentiate between the accreted and in-situ halo stars.

To test the predicted relations between ages and metallicities in different components of the Galaxy, we need to date stars in our sample.
Unfortunately, stellar ages are not a directly measurable quantity \citep[for a recent review, see][]{soderblom2010}.
A number of observables that correlate with age have been identified, such as stellar rotation \citep{barnes2007}, chromospheric activity \citep{mamajek2008}, or surface abundances \citep{ness2016}, but none of these empirical relations are applicable to all of the field stars.
Models of stellar evolution can relate the position of any star in the Hertzsprung--Russell diagram (HRD) and its internal structure to its age.
The latter is inferred from asteroseismic studies of stellar pulsations, and has so far been employed to date a few dozen of well observed stars \citep[e.g.,][]{keplerages}.
In the coming decade, asteroseismic dating will be expanded, but still limited to the brightest stars \citep{tess, plato}.
We expect the HRD age dating to be more easily applied to a larger sample of stars, and discuss it in more detail below.

Coeval stellar populations are routinely dated by comparison of their tracks in the HRD to theoretical isochrones \citep[e.g.,][]{sandage1970, chaboyer1998, dotter2007}, but isochrone dating of field stars is less straightforward.
Intrinsically, without the HRD positions of coeval companions, age estimates of field stars are very uncertain in evolutionary stages which keep stars at an approximately constant position in the HRD, such as the main sequence phase.
In addition, precisely measuring stellar distances, which are required to put a star on the HRD, as opposed to merely on a color-magnitude diagram, is observationally challenging.
However, if distances are known, stellar ages can be measured for stars in pre- or post-main sequence evolutionary stages.
\citet{gcs} measured ages and other intrinsic stellar parameters for thousands of nearby field stars by obtaining their absolute magnitudes from Hipparcos parallaxes, effective temperatures and metallicities from follow-up spectroscopy, and then reading off the age by interpolating theoretical isochrones in this three-dimensional space.
TGAS has already increased the sample of stars with known distances by an order of magnitude, and several groups are modeling the multi-band stellar photometry (and including spectroscopy when available) to provide constraints on their ages.
In such a procedure, ages of red giants are measured with a precision of $1-3$\;Gyr when only photometric data is available.
Including spectroscopically derived stellar parameters reduces the uncertainty in recovered ages to 1\;Gyr (P.~Cargile, private communication).
Most of the halo stars in our sample are giants, so if the bifurcation in the age--metallicity relation of halo stars exists at the level suggested by Latte, we will soon be able to detect it observationally.

% \section{Conclusions}


\vspace{0.5cm}
\emph{Acknowledgments:}
It is a pleasure to thank Andy Casey for providing a match of the RAVE-on catalog to TGAS, Yuan-Sen Ting for matching the APOGEE catalog to TGAS, Kim Venn, Rosy Wyse, Warren Brown, Elena D'Onghia, and Zoltan Haiman for insightful comments that shaped the progression of this project.

% software citations: matplotlib, numpy, scipy, gala, astropy

This paper was written in part at the 2016 NYC Gaia Sprint, hosted by the Center for Computational Astrophysics at the Simons Foundation in New York City.

This work has made use of data from the European Space Agency (ESA) mission {\it Gaia} (\url{http://www.cosmos.esa.int/gaia}), processed by the {\it Gaia} Data Processing and Analysis Consortium (DPAC, \url{http://www.cosmos.esa.int/web/gaia/dpac/consortium}). Funding for the DPAC has been provided by national institutions, in particular the institutions participating in the {\it Gaia} Multilateral Agreement.

Funding for RAVE has been provided by: the Australian Astronomical Observatory; the Leibniz-Institut fuer Astrophysik Potsdam (AIP); the Australian National University; the Australian Research Council; the French National Research Agency; the German Research Foundation (SPP 1177 and SFB 881); the European Research Council (ERC-StG 240271 Galactica); the Istituto Nazionale di Astrofisica at Padova; The Johns Hopkins University; the National Science Foundation of the USA (AST-0908326); the W. M. Keck foundation; the Macquarie University; the Netherlands Research School for Astronomy; the Natural Sciences and Engineering Research Council of Canada; the Slovenian Research Agency; the Swiss National Science Foundation; the Science \& Technology Facilities Council of the UK; Opticon; Strasbourg Observatory; and the Universities of Groningen, Heidelberg and Sydney.
The RAVE web site is at \url{https://www.rave-survey.org}.

\begin{figure*}
\begin{center}
\includegraphics[width=\textwidth]{../plots/paper/tdcontamination.pdf}
\caption{(Left) Probability contours of thick disk stars in the Toomre diagram in whole steps of standard deviation, $\sigma$ (orange lines).
All halo stars from our RAVEon--TGAS sample (metal-rich in light blue circles and metal-poor in dark blue squares) lie outside of the $3\;\sigma$ thick disk contour, but some are consistent with the thick disk at a $4\;\sigma$ level.
(Right) Probability for stars, identified in RAVEon--TGAS as part of the halo, of actually being a part of the thick disk.
Lines show cumulative fractions of halo stars as a function of this probability, with light blue for the metal-rich and dark blue for the metal-poor halo stars.
Only a small fraction of both halo components is expected to be a misclassified part of the thick disk (20\% of the metal-rich and 5\% of the metal-poor halo have a thick disk probability larger than 1\%, marked with a black vertical line).}
\label{fig:tdcont}
\end{center}
\end{figure*}

\bibliographystyle{apj}
\bibliography{apj-jour,mrich_halo}

\appendix{}
\section{Thick disk contamination}
\label{sec:tdcontamination}
The thick disk bridges the thin disk and the halo in both chemical abundances and kinematics.
Given how the metal-rich halo identified in this study has abundances consistent with the thick disk, in this section we quantify how different it is from the canonical thick disk kinematically.

As demonstrated by the toy model of the Solar neighborhood (\S\ref{sec:toymodel}), we expect some thick disk stars to enter our halo selection (Figure~\ref{fig:toy}, red points above the thick black line in the left panel).
We visualize the expected contamination levels in the left panel of Figure~\ref{fig:tdcont} by drawing the probability contours for the thick disk velocity ellipsoid \citep{bensby2003} in the Toomre diagram.
The successive contours enclose the parameter space occupied by the thick disk with probabilities of 68\%, 95\%, 99.7\% and 99.9\% (labeled as $1-4\;\sigma$ in Figure~\ref{fig:tdcont}).
Halo stars from our sample are shown as points, with metal-rich being represented by light blue circles and metal-poor by dark blue squares.
All of the halo stars are outside of the $3\;\sigma$ thick disk contour, or inconsistent with being a thick disk at the 99.7\% level, but 145 ($\approx25\%$) metal-rich and 25 ($\approx7\%$) metal-poor halo stars are inside the $4\;\sigma$ contour.
On the other hand, there are 453 metal-rich halo stars outside the $4\;\sigma$ thick disk contour, while only 16 thick disk stars are are expected in this region by the toy model.
Even though the velocity distributions of the halo and the thick disk are overlapping, and our halo sample may not be completely pure of the thick disk contaminants, there is a clear excess of stars with thick disk abundances beyond the canonical thick disk kinematics.

In the right panel of Figure~\ref{fig:tdcont} we quantify the probability of halo stars in our sample being a part of the thick disk, fully accounting for the observational uncertainties in all six observables.
The light blue line shows the cumulative fraction of metal-rich halo stars being a thick disk star at a given probability, while dark blue is the corresponding line for the metal-poor halo stars.
Less than 20\% of metal-rich halo stars have more than a percent probability (marked by a vertical black line) of being a misclassified thick disk star.
For the metal-poor halo, this fraction is even lower at 5\%.
The median probability of being a thick disk star is $\sim3\times10^{-4}$ and $\sim2\times10^{-6}$ for the metal-rich and the metal-poor halo, respectively, ruling out the thick disk interpretation of metal-rich stars identified in the local stellar halo.

So far, we have only considered the kinematic definition of a thick disk as measured by \citet{bensby2003}.
Studies based on different samples have arrived at slightly modified properties of a thick disk velocity ellipsoid \citep[e.g.,][]{soubiran2003, carollo2010}.
Furthermore, in a theoretical study of a thick disk formed in an idealized simulation, \citet{sb2009} noted that its velocity distribution in the Toomre diagram is much more asymmetric than the usually assumed Gaussian distribution function.
However, even this more extended definition of a thick disk does not encompass all of the metal-rich stars identified in the Solar neighborhood, excluding in particular stars on retrograde orbits with high $V_{XZ}$.
Assuming a different distribution function for the thick disk changes the inferred contamination levels in our halo sample in detail, but no disk-like distribution explains stars on very retrograde, warm orbits, where some of the metal-rich stars from our sample are found.

\end{document}

